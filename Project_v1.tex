% Options for packages loaded elsewhere
\PassOptionsToPackage{unicode}{hyperref}
\PassOptionsToPackage{hyphens}{url}
%
\documentclass[
]{article}
\title{Métodos estadísticos aplicados al baloncesto}
\usepackage{etoolbox}
\makeatletter
\providecommand{\subtitle}[1]{% add subtitle to \maketitle
  \apptocmd{\@title}{\par {\large #1 \par}}{}{}
}
\makeatother
\subtitle{Treball Fi de Grau d'Estadística Apliacada}
\author{Paula Moreno}
\date{21/1/2022}

\usepackage{amsmath,amssymb}
\usepackage{lmodern}
\usepackage{iftex}
\ifPDFTeX
  \usepackage[T1]{fontenc}
  \usepackage[utf8]{inputenc}
  \usepackage{textcomp} % provide euro and other symbols
\else % if luatex or xetex
  \usepackage{unicode-math}
  \defaultfontfeatures{Scale=MatchLowercase}
  \defaultfontfeatures[\rmfamily]{Ligatures=TeX,Scale=1}
\fi
% Use upquote if available, for straight quotes in verbatim environments
\IfFileExists{upquote.sty}{\usepackage{upquote}}{}
\IfFileExists{microtype.sty}{% use microtype if available
  \usepackage[]{microtype}
  \UseMicrotypeSet[protrusion]{basicmath} % disable protrusion for tt fonts
}{}
\makeatletter
\@ifundefined{KOMAClassName}{% if non-KOMA class
  \IfFileExists{parskip.sty}{%
    \usepackage{parskip}
  }{% else
    \setlength{\parindent}{0pt}
    \setlength{\parskip}{6pt plus 2pt minus 1pt}}
}{% if KOMA class
  \KOMAoptions{parskip=half}}
\makeatother
\usepackage{xcolor}
\IfFileExists{xurl.sty}{\usepackage{xurl}}{} % add URL line breaks if available
\IfFileExists{bookmark.sty}{\usepackage{bookmark}}{\usepackage{hyperref}}
\hypersetup{
  pdftitle={Métodos estadísticos aplicados al baloncesto},
  pdfauthor={Paula Moreno},
  hidelinks,
  pdfcreator={LaTeX via pandoc}}
\urlstyle{same} % disable monospaced font for URLs
\usepackage[margin=1in]{geometry}
\usepackage{color}
\usepackage{fancyvrb}
\newcommand{\VerbBar}{|}
\newcommand{\VERB}{\Verb[commandchars=\\\{\}]}
\DefineVerbatimEnvironment{Highlighting}{Verbatim}{commandchars=\\\{\}}
% Add ',fontsize=\small' for more characters per line
\usepackage{framed}
\definecolor{shadecolor}{RGB}{248,248,248}
\newenvironment{Shaded}{\begin{snugshade}}{\end{snugshade}}
\newcommand{\AlertTok}[1]{\textcolor[rgb]{0.94,0.16,0.16}{#1}}
\newcommand{\AnnotationTok}[1]{\textcolor[rgb]{0.56,0.35,0.01}{\textbf{\textit{#1}}}}
\newcommand{\AttributeTok}[1]{\textcolor[rgb]{0.77,0.63,0.00}{#1}}
\newcommand{\BaseNTok}[1]{\textcolor[rgb]{0.00,0.00,0.81}{#1}}
\newcommand{\BuiltInTok}[1]{#1}
\newcommand{\CharTok}[1]{\textcolor[rgb]{0.31,0.60,0.02}{#1}}
\newcommand{\CommentTok}[1]{\textcolor[rgb]{0.56,0.35,0.01}{\textit{#1}}}
\newcommand{\CommentVarTok}[1]{\textcolor[rgb]{0.56,0.35,0.01}{\textbf{\textit{#1}}}}
\newcommand{\ConstantTok}[1]{\textcolor[rgb]{0.00,0.00,0.00}{#1}}
\newcommand{\ControlFlowTok}[1]{\textcolor[rgb]{0.13,0.29,0.53}{\textbf{#1}}}
\newcommand{\DataTypeTok}[1]{\textcolor[rgb]{0.13,0.29,0.53}{#1}}
\newcommand{\DecValTok}[1]{\textcolor[rgb]{0.00,0.00,0.81}{#1}}
\newcommand{\DocumentationTok}[1]{\textcolor[rgb]{0.56,0.35,0.01}{\textbf{\textit{#1}}}}
\newcommand{\ErrorTok}[1]{\textcolor[rgb]{0.64,0.00,0.00}{\textbf{#1}}}
\newcommand{\ExtensionTok}[1]{#1}
\newcommand{\FloatTok}[1]{\textcolor[rgb]{0.00,0.00,0.81}{#1}}
\newcommand{\FunctionTok}[1]{\textcolor[rgb]{0.00,0.00,0.00}{#1}}
\newcommand{\ImportTok}[1]{#1}
\newcommand{\InformationTok}[1]{\textcolor[rgb]{0.56,0.35,0.01}{\textbf{\textit{#1}}}}
\newcommand{\KeywordTok}[1]{\textcolor[rgb]{0.13,0.29,0.53}{\textbf{#1}}}
\newcommand{\NormalTok}[1]{#1}
\newcommand{\OperatorTok}[1]{\textcolor[rgb]{0.81,0.36,0.00}{\textbf{#1}}}
\newcommand{\OtherTok}[1]{\textcolor[rgb]{0.56,0.35,0.01}{#1}}
\newcommand{\PreprocessorTok}[1]{\textcolor[rgb]{0.56,0.35,0.01}{\textit{#1}}}
\newcommand{\RegionMarkerTok}[1]{#1}
\newcommand{\SpecialCharTok}[1]{\textcolor[rgb]{0.00,0.00,0.00}{#1}}
\newcommand{\SpecialStringTok}[1]{\textcolor[rgb]{0.31,0.60,0.02}{#1}}
\newcommand{\StringTok}[1]{\textcolor[rgb]{0.31,0.60,0.02}{#1}}
\newcommand{\VariableTok}[1]{\textcolor[rgb]{0.00,0.00,0.00}{#1}}
\newcommand{\VerbatimStringTok}[1]{\textcolor[rgb]{0.31,0.60,0.02}{#1}}
\newcommand{\WarningTok}[1]{\textcolor[rgb]{0.56,0.35,0.01}{\textbf{\textit{#1}}}}
\usepackage{graphicx}
\makeatletter
\def\maxwidth{\ifdim\Gin@nat@width>\linewidth\linewidth\else\Gin@nat@width\fi}
\def\maxheight{\ifdim\Gin@nat@height>\textheight\textheight\else\Gin@nat@height\fi}
\makeatother
% Scale images if necessary, so that they will not overflow the page
% margins by default, and it is still possible to overwrite the defaults
% using explicit options in \includegraphics[width, height, ...]{}
\setkeys{Gin}{width=\maxwidth,height=\maxheight,keepaspectratio}
% Set default figure placement to htbp
\makeatletter
\def\fps@figure{htbp}
\makeatother
\setlength{\emergencystretch}{3em} % prevent overfull lines
\providecommand{\tightlist}{%
  \setlength{\itemsep}{0pt}\setlength{\parskip}{0pt}}
\setcounter{secnumdepth}{-\maxdimen} % remove section numbering
\ifLuaTeX
  \usepackage{selnolig}  % disable illegal ligatures
\fi

\begin{document}
\maketitle

\hypertarget{abstract}{%
\section{Abstract}\label{abstract}}

Hoy en día, el deporte es un hobby muy popular por todo el mundo. Des de
pequeños, la gran mayoría de niños practican algún tipo de deporte,
especialmente aquellos que son de equipo. Eso nos lleva a queres saber
más del deporte, más detalles, más información. Nos entra la curiosidad
de ``¿quién es el mejor jugador?'', ``Qué equipo és mejor?'' o incluso
intentar prevenir qué equipo ganará segun sus resultados anteriores. Y
gracias a los avances tecnológicos, cada vez se nos facilita más poder
seguir un deporte des de casa, ver la estadística de los deportistas e
incluso hay plataformas o juegos que nos permiten ser, de manera
virtual, managers de los clubs y, por lo tanto, nos facilitan mucha
información que antes era más difícil de saber.

Eso hace que, de manera progresiva también mejore el estudio y el
análisis de cada deporte, y cada vez sea más específica para cada
deporte, implementando nuevos recursos para mejorar los resultados.

En este trabajo estudiaremos más a fondo el Baloncesto, el segundo
deporte más popular de Europa (solo superado por el futbol), y el cual
yo practico des de los 4 años. En especial nos centraremos, en el
Baloncesto profesional Europeo, de donde podemos obtener más datos.

Este trabajo surgió del constante pensamiento de que los análisis
actuales que se hacen en este deporte en Europa son bastante pobres a
nivel informativo, ya que se basan en conceptos muy básicos. Para que
nos hagamos una idea, el estadístico por preferencia es el llamado
``Valoración'' y que se originó en 1991 (hace 30 años) y des de entonces
nunca se ha modificado.

Es por eso que, considero que actualmente los análisis que se hacen de
este deporte necesitan una actualización para llegar a informar de todos
aquellos datos que hoy en día si se pueden recoger gracias a los avances
tecnológicos.

\hypertarget{contenido}{%
\section{Contenido}\label{contenido}}

\begin{enumerate}
\def\labelenumi{\arabic{enumi}.}
\setcounter{enumi}{-1}
\item
  Recursos informaticos nuevos que se han utilizado para hacer este
  trebajo: GitHub. Breve explicacion de qué es, como funciona y ventajas
  que tiene.
\item
  Introducción al baloncesto y breve explicación de cómo se juega
  (historia, conceptos importantes de conocer, y metodologia del juego)
\item
  Explicación de la Estadística Actual (formulas, definiciones)
\item
  Nuevos Analisis (breve explicacion de la nueva forma de recoger datos
  de los jugadores, y luego el mas/menos ajustado, Gini\ldots)
\item
  Analisis de los datos Temporada 2018-2019
\end{enumerate}

\hypertarget{recursos-informaticos}{%
\section{Recursos Informaticos}\label{recursos-informaticos}}

Para realizar este trabajo, mi tutor Sergio Olmos, me recomendo utilizar
GitHub porque era una manera de poder compartir mi proyecto de Markdown
con mis tutores de manera constante. Requeria trabajar con el proyecto
publicado en dicha plataforma y de manera automática, cuando yo
modificara mi archivo, ellos podrian ver este cambio al momento. Yo
desconocia totalmente de este espacio, por lo que una parte de mi
trabajo era aprender a trabajar en GitHub.

\hypertarget{quuxe9-es-github}{%
\subsubsection{¿Qué es GitHub?}\label{quuxe9-es-github}}

GitHub es una plataforma de alojamiento, propiedad de Microsoft, que
ofrece a los desarrolladores la posibilidad de crear repositorios de
código y guardarlos en la nube de forma segura, usando un sistema de
control de versiones, llamado Git.

Como he comentado, facilita la organización de proyectos y permite la
colaboración de varios desarrolladores en tiempo real. Es decir, nos
permite centralizar el contenido del repositorio para poder colaborar
con los otros miembros de nuestra organización.

GitHub esta basada en el sistema de control de versiones distribuida de
Git, por lo que se puede contar con sus funciones y herramientas, aunque
GitHub ofrece varias opciones adicionales y su interfaz es mucho más
fácil de manejar, por lo que no es absolutamente necesario que las
personas que lo usan tengan un gran conocimiento técnico.

\hypertarget{ventajas}{%
\subsubsection{Ventajas}\label{ventajas}}

Existe un gran número de razones que convierten a GitHub en una gran
opción para el control y gestión de tus proyectos de código. Aquí
algunas de ellas:

\begin{itemize}
\tightlist
\item
  GitHub permite que alojemos proyectos en repositorios de forma
  gratuita
\item
  Te brinda la posibilidad de personalizar tu perfil en la plataforma
\item
  Los repositorios son públicos por defecto. Sin embargo, GitHub te
  permite también alojar tus proyectos de forma privada
\item
  Puedes crear y compartir páginas web estáticas con GitHub Pages
\item
  Facilita compartir tus proyectos de una forma mucho más fácil y crear
  un portafolio
\item
  Te permite colaborar para mejorar los proyectos de otros y a otros
  mejorar o aportar a los tuyos
\item
  Ayuda reducir significativamente los errores humanos y escribir tu
  código más rápido con GitHub Copilot
\item
  Te da control de versiones, una herramienta muy útil.
\end{itemize}

\hypertarget{quuxe9-es-el-control-de-versiones}{%
\subsubsection{¿Qué es el control de
versiones?}\label{quuxe9-es-el-control-de-versiones}}

Se le llama control de versiones a la administración de los cambios que
se realizan sobre los elementos o la configuración de algún proyecto. En
otras palabras, el control de versiones sirve para conocer y autorizar
los cambios que realicen los colaboradores en tu proyecto, guardando
información de qué incluyen los cambios y cuándo se hicieron. Este
control comienza con una versión básica del documento y luego va
guardando los cambios que se realicen a lo largo del proyecto.

El control de versiones es una herramienta valiosísima, pues con ella
puedes tener acceso a las versiones anteriores de tu proyecto si es que
en algún momento no llega a funcionar de forma correcta.

\hypertarget{quuxe9-es-git}{%
\subsubsection{¿Qué es Git?}\label{quuxe9-es-git}}

Git es un software de control de versiones diseñado por Linus Torvalds,
pensando en la eficiencia, la confiabilidad y compatibilidad del
mantenimiento de versiones de aplicaciones cuando estas tienen un gran
número de archivos de código fuente.

\hypertarget{diferencias-git-vs-github}{%
\paragraph{Diferencias Git vs GitHub}\label{diferencias-git-vs-github}}

Entonces, ¿qué diferencia Git de GitHub?. La principal diferencia es que
Git es un sistema que permite establecer un control de versiones,
mientras que GitHub es una plataforma que ofrece un grupo de funciones
que facilitan el uso de Git y la colaboración en tiempo real, así como
el almacenamiento en la nube.

\hypertarget{el-baloncesto}{%
\section{El baloncesto}\label{el-baloncesto}}

\hypertarget{historia}{%
\subsection{Historia}\label{historia}}

El baloncesto es un deporte de equipo que se originó en 1891, por James
Naismith, profesor de educación física en la escuela, que buscaba idear
un deporte que sus alumnos pudieran practicar bajo techo, pues los duros
inviernos en Nueva Inglaterra dificultaban la realización de ejercicio
al aire libre. Se basó en un juego de su infancia que consisitia en
alcanzar un objectivo con una piedra, e intento encargar 50 cajas para
repartirlas por el gimnasio para que los alumnos encestaran en ellas.
Pero al final solo pudieron obtener dos cajas de melocotones, que
decidió colgar encima de las barandillas de la galería superior que
rodeaba el gimnasio, a una determinada altura.

Inicialmente se jugaba con 9 jugadores por equipo en cancha (tenia un
total de 18 alumnos y para que todos estubieran involucrados decidió que
todos jugaran a la vez) e ideó un total de trece reglas para que el
juego pudiera fluir sin problemas:

\begin{enumerate}
\def\labelenumi{\arabic{enumi}.}
\tightlist
\item
  El balón puede ser lanzado en cualquier dirección, con una o dos
  manos.
\item
  El balón puede ser palmeado/golpeado en cualquier dirección, con una o
  las dos manos (nunca con el puño o mano cerrada).
\item
  Los jugadores no podrán correr con el balón. Deberán pasarlo incluso
  desde otro lugar en el que lo cogieron, se concederá una relativa
  tolerancia al jugador que en plena carrera reciba el balón y deba
  pararse.
\item
  El balón debe llevarse en las manos o entre ellas. Los brazos o el
  cuerpo no se deben usar para sostenerlo en ningún caso.
\item
  Está prohibido cargar con el hombro contra un adversario, así como
  agarrar, empujar, poner la zancadilla o golpear de manera alguna al
  oponente. Toda infracción a esta regla por parte de cualquier jugador
  se considerará una falta y en caso de reincidencia, el infractor será
  eliminado hasta que se consiga un nuevo cesto. Si la intención al
  golpear es evidente, el jugador será eliminado por el resto del
  partido y no podrá ser reemplazado.
\item
  Golpear con el puño el balón es falta, al ser violación de las reglas
  2 y 4, sancionándose del mismo modo que la regla 5.
\item
  Si cualquiera de los equipos hace tres faltas personales consecutivas,
  se contabilizará una canasta para el equipo contrario (consecutivas
  significa que durante ese tiempo el oponente no haya cometido ninguna
  falta).
\item
  Se contará canasta cuando el balón sea lanzado, golpeado o palmado
  desde el suelo hasta la cesta y se quede en ella, los defensores nunca
  tienen que tocar el balón o dificulten la canasta. Si el balón se
  queda en el borde de la cesta sin llegar a entrar y el oponente mueve
  la canasta, se contabilizará como punto.
\item
  Cuando el balón salga fuera del campo de juego, volverá al campo. La
  primera persona que lo toque lo lanzará al campo de juego. En caso de
  discusión el árbitro (auxiliar) realizará un salto entre dos. El que
  saca dispone de cinco segundos para hacerlo; si retiene el balón más
  tiempo, el balón pasará al equipo contrario. Si cualquiera de los
  equipos persiste en retrasar el juego, el árbitro auxiliar le señalará
  falta.
\item
  El árbitro auxiliar será el juez que anote las faltas personales y
  avisará al árbitro principal cuando se cometan tres faltas
  consecutivas. Podrá descalificar a los jugadores según lo establecido
  en la regla número 5.
\item
  El árbitro principal juzgará lo que se refiere al balón y determinará
  cuándo éste está en juego o ha salido fuera, a qué equipo pertenece,
  además de llevar el control del tiempo. Decidirá cuándo se ha marcado
  un tanto y contabilizará las canastas y asimismo realizará las
  obligaciones habituales de un árbitro.
\item
  El partido constará de dos partes de 15 minutos, con 5 minutos de
  descanso entre las mismas.
\item
  El equipo que obtenga el mayor número de cestos en ese espacio de
  tiempo será declarado ganador. En caso de empate, si los capitanes
  acuerdan hacerlo, el partido se podrá continuar hasta que se marque
  una canasta.
\end{enumerate}

Con el paso de los años, este deporte que empezó como actividad de
colegio, ha ido evolucionando mucho, añadiendo más reglas, conceptos
nuevos, cambios en el número de jugadores, se ha determinado tiempos de
juego, hay nuevos puntuages segun la distancia des de donde se anote la
canasta, etc.

Actualmente, las normas más básicas de este deporte son:

\begin{itemize}
\tightlist
\item
  En las ligas superiores, hay un total de 4 cuartos de 10 minutos y
  pueden estar en pista 5 jugadores por equipo.
\item
  No te puedes desplazar con la pelota en las manos, es obligatorio
  botar con una mano (sino será una infracción y conllevará la perdida
  de pelota y saque de banda del equipo rival).
\item
  Cada jugador puede realizar hasta un total de 5 faltas, que sera
  penalizado con un saque de banda o con un tiro libre (dependera de la
  situación). El jugador que realizé 5 faltas será expulsado del
  partido.
\item
  El objetivo es encestar el máximo de puntos posibles, teniendo en
  cuenta que pueden sumar 1, 2 o 3 puntos.
\end{itemize}

\hypertarget{conceptos-y-definiciones-buxe1sicos-del-baloncesto}{%
\subsection{Conceptos y definiciones básicos del
baloncesto:}\label{conceptos-y-definiciones-buxe1sicos-del-baloncesto}}

Para que podamos entender a que nos referimos en estre trabajo, es
necesario comprender unos conceptos básicos de vocabulario. Tendremos en
cuenta los conceptos que se necesitan para realizar la valoración del
jugador y/o del equipo que se utilizan en las estadísticas federadas.

\begin{itemize}
\item
  Falta: Acción en la que un defensor bloquea el avance de su rival sin
  tener control de balón o de manera no reglamentaria (empujar,
  agarrar\ldots)
\item
  Perdidas de balón: cuando un equipo pierde el control del balón y pasa
  a ser del equipo rival.
\item
  Rebotes: Recuperación de pelota después de que el tiro sea realizado
  pero no haya encestado.
\item
  Recuperación de balón: Cuando un equipo consigue robar el balón al
  equipo rival.
\item
  Asistencia: Es un pase a un jugador que se encuentra en una posición
  de ventaja o que le ayuda a conseguir una canasta sin realizar ningun
  bote.
\item
  Tapón: Bloqueo de un tiro en el aire.
\item
  Más/menos: Estadístico que se determina observando la diferencia de
  puntos que se realizan durante los minutos que el jugador esta
  jugando.
\item
  Valoración: Estadístico que en general, engloba todo lo que pasa en el
  partido de manera individual. Cuanto más positivo, mejor. Sigue la
  siguiente formula:

  \[V = (Puntos + Rebotes + Asistencias + Robos + Tapones + Faltas Recibidas) - (Tiros de Campo Fallados + Tiros Libres Fallados + Tapones Recibidos + Pérdidas + Faltas Realizadas) \]
\end{itemize}

\begin{center}\rule{0.5\linewidth}{0.5pt}\end{center}

\hypertarget{bibliografia}{%
\section{Bibliografia}\label{bibliografia}}

Sport in Europe (Wikipedia):
\url{https://en.wikipedia.org/wiki/Sport_in_Europe}

Valoración (Wikipedia):
\url{https://en.wikipedia.org/wiki/Performance_Index_Rating}

Git (Wikipedia): \url{https://es.wikipedia.org/wiki/Git}

GitHub (Wikipedia): \url{https://es.wikipedia.org/wiki/GitHub}

Baloncesto (Wikipedia): \url{https://es.wikipedia.org/wiki/Baloncesto}

Valoración (Wikipedia):
\url{https://es.wikipedia.org/wiki/Valoraci\%C3\%B3n_(baloncesto)}

\begin{center}\rule{0.5\linewidth}{0.5pt}\end{center}

\hypertarget{datos}{%
\section{Datos}\label{datos}}

\begin{Shaded}
\begin{Highlighting}[]
\FunctionTok{library}\NormalTok{(readr)}
\NormalTok{pbp2018 }\OtherTok{\textless{}{-}} \FunctionTok{read.csv}\NormalTok{(}\AttributeTok{file=}\StringTok{"pbp2018.csv"}\NormalTok{, }\AttributeTok{head=}\ConstantTok{TRUE}\NormalTok{, }\AttributeTok{sep=}\StringTok{","}\NormalTok{)}
\CommentTok{\#View(pbp2018)}
\end{Highlighting}
\end{Shaded}

\hypertarget{paquetes}{%
\subsubsection{Paquetes}\label{paquetes}}

\begin{Shaded}
\begin{Highlighting}[]
\FunctionTok{library}\NormalTok{(dplyr)}
\FunctionTok{library}\NormalTok{(ggplot2)}
\end{Highlighting}
\end{Shaded}

\hypertarget{copy-and-paste---areglarlo}{%
\paragraph{COPY AND PASTE -
Areglarlo!}\label{copy-and-paste---areglarlo}}

\hypertarget{alineaciones-con-jugadores-repetidos}{%
\subsubsection{Alineaciones con jugadores
repetidos}\label{alineaciones-con-jugadores-repetidos}}

\hypertarget{paquetes-1}{%
\subsection{Paquetes}\label{paquetes-1}}

\begin{Shaded}
\begin{Highlighting}[]
\FunctionTok{library}\NormalTok{(here)}
\FunctionTok{library}\NormalTok{(tidyverse)}
\end{Highlighting}
\end{Shaded}

\hypertarget{problema}{%
\subsection{Problema}\label{problema}}

Algunos registros en \texttt{pbp\_2018.csv} contenían alineaciones con
un mismo jugador que aperece dos veces.

Tras inspeccionar la función del paquete eurolig que obtiene las
alineaciones de estos datos, descubrí que se trataba de un error en la
función que asigna las alineaciones en las situaciones en las que hay
cambios de jugadores cuando hay tiros libres. Solo afectaba a las
variables \texttt{away\_player4} y \texttt{away\_player5}.

\begin{Shaded}
\begin{Highlighting}[]
\NormalTok{pbp\_2018 }\OtherTok{\textless{}{-}} \FunctionTok{read\_csv}\NormalTok{(}\StringTok{"pbp2018.csv"}\NormalTok{)}
\end{Highlighting}
\end{Shaded}

\begin{verbatim}
## Warning: One or more parsing issues, see `problems()` for details
\end{verbatim}

\begin{Shaded}
\begin{Highlighting}[]
\DocumentationTok{\#\# Check how many rows are affected by this}
\NormalTok{bad\_lineups }\OtherTok{\textless{}{-}}\NormalTok{ pbp\_2018 }\SpecialCharTok{\%\textgreater{}\%}
  \FunctionTok{select}\NormalTok{(}\FunctionTok{matches}\NormalTok{(}\StringTok{"\_player[1{-}5]"}\NormalTok{)) }\SpecialCharTok{\%\textgreater{}\%}
  \FunctionTok{apply}\NormalTok{(}\DecValTok{1}\NormalTok{, }\ControlFlowTok{function}\NormalTok{(x) }\FunctionTok{max}\NormalTok{(}\FunctionTok{table}\NormalTok{(x)) }\SpecialCharTok{\textgreater{}} \DecValTok{1}\NormalTok{)}
\NormalTok{pbp\_bad }\OtherTok{\textless{}{-}}\NormalTok{ pbp\_2018 }\SpecialCharTok{\%\textgreater{}\%}
  \FunctionTok{filter}\NormalTok{(bad\_lineups)}
\NormalTok{pbp\_bad }\SpecialCharTok{\%\textgreater{}\%}
  \FunctionTok{select}\NormalTok{(season, game\_code, play\_number, play\_type, away\_player4, away\_player5)}
\end{Highlighting}
\end{Shaded}

\begin{verbatim}
## # A tibble: 10,260 x 6
##    season game_code play_number play_type away_player4 away_player5
##     <dbl>     <dbl>       <dbl> <chr>     <chr>        <chr>       
##  1   2018         2          93 IN        TOMIC, ANTE  TOMIC, ANTE 
##  2   2018         2          94 OUT       TOMIC, ANTE  TOMIC, ANTE 
##  3   2018         2          95 OUT       TOMIC, ANTE  TOMIC, ANTE 
##  4   2018         2          96 IN        TOMIC, ANTE  TOMIC, ANTE 
##  5   2018         2          97 OUT       TOMIC, ANTE  TOMIC, ANTE 
##  6   2018         2          98 IN        TOMIC, ANTE  TOMIC, ANTE 
##  7   2018         2          99 OUT       TOMIC, ANTE  TOMIC, ANTE 
##  8   2018         2         100 FTM       TOMIC, ANTE  TOMIC, ANTE 
##  9   2018         2         101 FTM       TOMIC, ANTE  TOMIC, ANTE 
## 10   2018         2         116 AST       KURIC, KYLE  KURIC, KYLE 
## # ... with 10,250 more rows
\end{verbatim}

\hypertarget{soluciuxf3n}{%
\subsection{Solución}\label{soluciuxf3n}}

He creado unas funciones en \href{fix-lineups.R}{\texttt{fix-lineups.R}}
para corregir este error en los datos que ya tenemos.

\begin{Shaded}
\begin{Highlighting}[]
\FunctionTok{source}\NormalTok{(}\FunctionTok{here}\NormalTok{(}\StringTok{"R"}\NormalTok{, }\StringTok{"fix{-}lineups.R"}\NormalTok{))}
\DocumentationTok{\#\# Function fix\_lineups() only takes data from a single game,}
\DocumentationTok{\#\# so I split the data and apply the function to each splitted data frame.}
\NormalTok{pbp\_2018\_fixed }\OtherTok{\textless{}{-}} \FunctionTok{split}\NormalTok{(pbp\_2018, }\FunctionTok{factor}\NormalTok{(pbp\_2018}\SpecialCharTok{$}\NormalTok{game\_code)) }\SpecialCharTok{\%\textgreater{}\%}
  \FunctionTok{map\_df}\NormalTok{(fix\_lineups)}
\DocumentationTok{\#\# Check that this has been fixed}
\NormalTok{pbp\_2018\_fixed }\SpecialCharTok{\%\textgreater{}\%}
  \FunctionTok{select}\NormalTok{(}\FunctionTok{matches}\NormalTok{(}\StringTok{"\_player[1{-}5]"}\NormalTok{)) }\SpecialCharTok{\%\textgreater{}\%}
  \FunctionTok{apply}\NormalTok{(}\DecValTok{1}\NormalTok{, }\ControlFlowTok{function}\NormalTok{(x) }\FunctionTok{max}\NormalTok{(}\FunctionTok{table}\NormalTok{(x)) }\SpecialCharTok{\textgreater{}} \DecValTok{1}\NormalTok{) }\SpecialCharTok{\%\textgreater{}\%}
  \FunctionTok{sum}\NormalTok{()}
\end{Highlighting}
\end{Shaded}

\begin{verbatim}
## [1] 0
\end{verbatim}

Finalmente escribo los datos corregidos en
\texttt{data/pbp\_2018\_fixed.csv}.

\begin{Shaded}
\begin{Highlighting}[]
\CommentTok{\#write\_csv(pbp\_2018\_fixed, here("data", "pbp\_2018\_fixed.csv"))}

\CommentTok{\#pbp2018\_f \textless{}{-} read.csv(file="pbp\_2018\_fixed.csv", head=TRUE, sep=",")}
\CommentTok{\#View(pbp2018\_f)}
\end{Highlighting}
\end{Shaded}

\begin{center}\rule{0.5\linewidth}{0.5pt}\end{center}

\hypertarget{masmenos}{%
\section{MAS/MENOS}\label{masmenos}}

\hypertarget{code-siguiendo-enlace-sobre-masmenos-ajustado-en-r-season-2018}{%
\section{Code siguiendo enlace sobre: MAS/MENOS AJUSTADO en R (season
2018)}\label{code-siguiendo-enlace-sobre-masmenos-ajustado-en-r-season-2018}}

\begin{Shaded}
\begin{Highlighting}[]
\CommentTok{\# Lista de jugadores:}
\NormalTok{players\_rep }\OtherTok{\textless{}{-}}\NormalTok{ pbp\_2018\_fixed}\SpecialCharTok{$}\NormalTok{player\_name}
\NormalTok{players }\OtherTok{\textless{}{-}} \FunctionTok{unique}\NormalTok{(players\_rep)}

\CommentTok{\# Matriz Stint y vectores respuesta:}
\NormalTok{stints }\OtherTok{\textless{}{-}} \FunctionTok{as.data.frame}\NormalTok{(}\FunctionTok{matrix}\NormalTok{( ,}\DecValTok{0}\NormalTok{,}\FunctionTok{length}\NormalTok{(players)))}
\FunctionTok{names}\NormalTok{(stints) }\OtherTok{\textless{}{-}}\NormalTok{ players}
\NormalTok{stintCount }\OtherTok{=} \DecValTok{1}

\CommentTok{\# Acciones que termina una posesion de pelota:}
\DocumentationTok{\#\# Canastas encestada (2/3/tiro libre), rebote defensivo, pelota recuperada}
\NormalTok{shots }\OtherTok{\textless{}{-}}\NormalTok{ pbp\_2018\_fixed[}\FunctionTok{which}\NormalTok{(pbp\_2018\_fixed}\SpecialCharTok{$}\NormalTok{play\_type }\SpecialCharTok{==} \StringTok{"2FGM"} \SpecialCharTok{|}\NormalTok{ pbp\_2018\_fixed}\SpecialCharTok{$}\NormalTok{play\_type }\SpecialCharTok{==} \StringTok{"3FGM"} \SpecialCharTok{|}\NormalTok{ pbp\_2018\_fixed}\SpecialCharTok{$}\NormalTok{play\_type }\SpecialCharTok{==} \StringTok{"FTM"} \SpecialCharTok{|}\NormalTok{ pbp\_2018\_fixed}\SpecialCharTok{$}\NormalTok{play\_type }\SpecialCharTok{==} \StringTok{"DRB"} \SpecialCharTok{|}\NormalTok{ pbp\_2018\_fixed}\SpecialCharTok{$}\NormalTok{play\_type }\SpecialCharTok{==} \StringTok{"TOV"}\NormalTok{),]}

\FunctionTok{View}\NormalTok{(shots)}

\NormalTok{awayplayersStart }\OtherTok{\textless{}{-}} \FunctionTok{levels}\NormalTok{(}\FunctionTok{droplevels}\NormalTok{(}\FunctionTok{as.factor}\NormalTok{(}\FunctionTok{unique}\NormalTok{(}\FunctionTok{unlist}\NormalTok{(shots[}\DecValTok{30}\NormalTok{,}\DecValTok{24}\SpecialCharTok{:}\DecValTok{28}\NormalTok{])))))}
\NormalTok{homeplayersStart }\OtherTok{\textless{}{-}} \FunctionTok{levels}\NormalTok{(}\FunctionTok{droplevels}\NormalTok{(}\FunctionTok{as.factor}\NormalTok{(}\FunctionTok{unique}\NormalTok{(}\FunctionTok{unlist}\NormalTok{(shots[}\DecValTok{1}\NormalTok{,}\DecValTok{19}\SpecialCharTok{:}\DecValTok{23}\NormalTok{])))))}

\NormalTok{num\_shots }\OtherTok{\textless{}{-}} \FunctionTok{dim}\NormalTok{(shots)[}\DecValTok{1}\NormalTok{]}

\NormalTok{awayplayers}\OtherTok{\textless{}{-}}\FunctionTok{matrix}\NormalTok{(,num\_shots,}\DecValTok{5}\NormalTok{)}
\NormalTok{homeplayers}\OtherTok{\textless{}{-}}\FunctionTok{matrix}\NormalTok{(,num\_shots,}\DecValTok{5}\NormalTok{)}

\DocumentationTok{\#\#\# ERROR!!}
\ControlFlowTok{for}\NormalTok{ (i }\ControlFlowTok{in} \DecValTok{1}\SpecialCharTok{:}\NormalTok{num\_shots) \{}
\NormalTok{  awayplayers[i,] }\OtherTok{\textless{}{-}} \FunctionTok{levels}\NormalTok{(}\FunctionTok{droplevels}\NormalTok{(}\FunctionTok{as.factor}\NormalTok{(}\FunctionTok{unique}\NormalTok{(}\FunctionTok{unlist}\NormalTok{(shots[i,}\DecValTok{24}\SpecialCharTok{:}\DecValTok{28}\NormalTok{])))))}
\NormalTok{  homeplayers[i,] }\OtherTok{\textless{}{-}} \FunctionTok{levels}\NormalTok{(}\FunctionTok{droplevels}\NormalTok{(}\FunctionTok{as.factor}\NormalTok{(}\FunctionTok{unique}\NormalTok{(}\FunctionTok{unlist}\NormalTok{(shots[i,}\DecValTok{19}\SpecialCharTok{:}\DecValTok{23}\NormalTok{])))))}
\NormalTok{\}}

\FunctionTok{View}\NormalTok{(awayplayers)}

\NormalTok{bothHome }\OtherTok{\textless{}{-}}\NormalTok{ homeplayersStart }\SpecialCharTok{\%in\%}\NormalTok{ homeplayers}
\NormalTok{bothAway }\OtherTok{\textless{}{-}}\NormalTok{ awayplayersStart }\SpecialCharTok{\%in\%}\NormalTok{ awayplayers}

\ControlFlowTok{if}\NormalTok{(}\FunctionTok{all}\NormalTok{(bothHome) }\SpecialCharTok{==} \ConstantTok{TRUE} \SpecialCharTok{\&} \FunctionTok{all}\NormalTok{(bothAway)}\SpecialCharTok{==}\ConstantTok{TRUE}\NormalTok{)\{\}}
\end{Highlighting}
\end{Shaded}

\begin{verbatim}
## NULL
\end{verbatim}

\end{document}
