\documentclass[paper=a4, fontsize=9pt]{article}
\usepackage[utf8]{inputenc}

\usepackage[a4paper]{geometry}
\geometry{top=2cm, bottom=2cm, left=2cm, right=2cm}
\setlength{\parskip}{2mm}

\usepackage{lipsum}

\usepackage[spanish]{babel}										
\usepackage[protrusion=true,expansion=true]{microtype}		    % Better typography
\usepackage{amsmath,amsfonts,amsthm}					                % Math packages
\usepackage[pdftex]{graphicx}									                % Enable pdflatex
\usepackage[svgnames]{xcolor}									                % Enabling colors by their 'svgnames'
\usepackage[hang, small, labelfont=bf,up,textfont=it,up]{caption}	% Custom captions under/above floats
\usepackage{epstopdf}											  	                % Converts .eps to .pdf
\usepackage{subfig}												  	                % Subfigures
\usepackage{float}
\usepackage{booktabs}											  	                % Nicer tables
\usepackage{fix-cm}												  	                % Custom fontsizes
\usepackage{hyperref}                                         % Link Clicks


%opening
\title{\textbf{Métodos estadísticos aplicados al baloncesto}}
\date{}


\begin{document}

\maketitle

Alumna: Paula Moreno Blazquez 

Tutores: Sergio Olmos (tutor externo) y Jose Barreras (tutor UAB)

\begin{abstract}

Hoy en día, el deporte es un hobby muy popular por todo el mundo. Des de pequeños, los niños practican algún tipo de deporte, especialmente aquellos que son de equipo. Eso nos lleva a querer saber más del deporte, más detalles, más información. Nos entra la curiosidad de "¿quién es el mejor jugador?", "Qué equipo es mejor?", o incluso intentar prevenir qué equipo ganará según sus resultados anteriores. Y gracias a los avances tecnológicos e informáticos, cada vez se nos facilita más poder seguir un deporte des de casa, ver la estadística de los deportistas e incluso hay plataformas o juegos que nos permiten ser, de manera virtual, managers de los clubs y, por lo tanto, nos facilitan mucha información que antes era más difícil de saber.

Eso hace que, de manera progresiva, también mejore el estudio y el análisis de cada deporte, y cada vez sea más específica para cada deporte, implementando nuevos recursos para mejorar los resultados. Pero, ¿son lo suficientemente eficaces los análisis que se realizan actualmente en Europa? ¿O dichos análisis están anticuados y requieren de una actualización?

\end{abstract}

\section*{Objetivos}
En este TFG se plantea hacer un estudio más detallado del que se hace actualmente de los partidos de baloncesto en Europa, e implementaremos un análisis ya desarrollado fuera de Europa (en la NBA) que es más justo en el momento de repartir pesos a las acciones (tanto ofensivas como defensivas).

Además, también se trabajará con GitHub, una plataforma web para el desarrollo y uso compartido de proyectos de código que yo desconocía como usar y que parece muy interesante de utilizar.

\begin{thebibliography}{X}
	\bibitem{RAPM} \textsc{Joseph Sill}, \textit{Improved NBA Adjusted +/- Using Regularization and Out-of-Sample Testing}, PDF, 6 Marzo 2010.
	
	\bibitem{HappyGH} \textsc{Happy Git}, \textit{Let’s Git started}, url: \url{https://happygitwithr.com/index.html}, .
\end{thebibliography}


\end{document}
